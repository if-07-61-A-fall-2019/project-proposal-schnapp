\documentclass[12pt]{article}
\usepackage{geometry}                % See geometry.pdf to learn the layout options. There are lots.
\geometry{letterpaper}                   % ... or a4paper or a5paper or ... 
\usepackage{graphicx}
\usepackage{amssymb}
\usepackage{amsthm}
\usepackage{epstopdf}
\usepackage[utf8]{inputenc}
\usepackage[usenames,dvipsnames]{color}
\usepackage[table]{xcolor}
\usepackage{hyperref}
\
\DeclareGraphicsRule{.tif}{png}{.png}{`convert #1 `dirname #1`/`basename #1 .tif`.png}

\theoremstyle{definition}
\newtheorem{example}{Example}

\newenvironment{explanation}{%
   \setlength{\parindent}{0pt}
   \itshape
   \color{blue}
}{}

\newcommand{\projectname}{Schnapp 5}
\newcommand{\productname}{School Health Service Assistant}
\newcommand{\projectleader}{P. Litzlfellner}
\newcommand{\documentstatus}{In process}
%\newcommand{\documentstatus}{Submitted}
%\newcommand{\documentstatus}{Released}
\newcommand{\version}{V. 1.0}

\begin{document}
\begin{titlepage}
\begin{flushright}
\includegraphics[scale=.5]{htlleondinglogo.png}\\
\end{flushright}

\vspace{10em}

\begin{center}
{\Huge Project Proposal} \\[3em]
{\LARGE \productname} \\[3em]
\end{center}

\begin{flushleft}
\begin{tabular}{|l|l|}
\hline
Project Name & \projectname \\ \hline
Project Leader & \projectleader \\ \hline
Document state & \documentstatus \\ \hline
Version & \version \\ \hline
\end{tabular}
\end{flushleft}

\end{titlepage}
\section*{Revisions}
\begin{tabular}{|l|l|l|}
\hline
\cellcolor[gray]{0.5}\textcolor{white}{Date} & \cellcolor[gray]{0.5}\textcolor{white}{Author} & \cellcolor[gray]{0.5}\textcolor{white}{Change} \\ \hline
September 27, 2019&Patrick Litzlfellner &First version \\ \hline
\end{tabular}
\pagebreak

\tableofcontents
\pagebreak

\section{Introduction}



Schnapp ist ein Programm, das Kindern in der Unterstufe und in der Vorschule über bestimmte Deutschkompetenzen, wie zum Beispiel Buchstaben erkennen, wiedergeben oder das zuordnen von Zahlen, mit spielerischen Fragen getestet wird.
Diese Tests werden jedes Jahr am Schulanfang in den Volksschulen an die 1. Klässler geprüft.
Diese Daten werden alle in einer Datenbank gesammelt. 


\pagebreak

\section{Initial Situation}
\begin{explanation}
The initial situation presents the assessment of the actual situation of an organizational unit or the entire organization of an agency or company. Thus a need for action, which may lead to a product or system vision, is recognizable. The vision may be developed into a project idea. The need for action may be initiated by several project or system ideas.
The demonstration of capability gaps (i.e. the difference between the necessary planned capabilities and the actually existing capabilities) in a company or agency may clearly show an urgent need for action in order to increase the efficiency or reduce costs. This need for action is presented as product or system idea, leading frequently to a concrete project proposal. Correspondingly, the determination of the requirement to renew or improve a "technically obsolete" system (so-called "system regeneration") or the recognition of market chances for a new product or system may lead to a project idea. The applicable data must be developed for the project proposal.
Research programs or studies may also be the basis for project ideas; they will be concretized in a project proposal.

The basic question could be summarized in German as follows:
\begin{itemize}
	\item Die Ist-Fähigkeiten der Organisation (was können wir?)
	\item Die Soll-Fähigkeiten der Organisation (was wollen wir können?)
	\item Ein Soll-Ist-Fähigkeitenvergleich (wo liegen die Defizite?)
	\item Ein Fähigkeitsvergleich nach vorgegebenen Bewertungskriterien
\end{itemize}
\end{explanation}



Dieses Programm wird mit einem Prüfer und Kindern im Volkschulalter verwendet. Die Kinder gehen in ihrem eigenen Tempo durch die Aufgaben und erhalten Hilfe durch den Prüfer. Es gibt auch Aufgaben, die nur mit dem Prüfer erledigt werden können. Die Tests können mehrmals wiederholt werden und die Kinder bekommen die Zeit, die sie brauchen um die Tests abzuschließen. Wenn alle Aufgaben abgeschlossen wurden, werden die Ergebnisse auf einen Server hochgeladen. Auf dem Server werden dann die Ergebnisse aller Kinder miteinander verglichen.
Das Ergebnis dieses Tests soll sein, dass die Lehrer es einfacher haben die einzelnen Kinder zu fördern, in den Bereichen in denen sie noch Schwierigkeiten haben.
Aktuell haben wir 11 Aufgaben mit passenden Audios und Bildern zu jeder Übung.
Wir möchten das Programm noch weiter verbessern, sodass es den Kindern besser helfen kann.
Wir wollen eine bessere Förderung der Kinder durch die Lehrkräfte erreichen und ihnen so eine bessere Zukunft ermöglichen.
Die Tests werden an jeder Klasse der Volksschule durchgeführt. So sieht man jedes Jahr, wie weit jedes Kind ist und wo es eventuell noch Förderungen benötigt.
Das Programm wird auf eine iPad ausgeführt oder am PC. Man benötigt Erfahrung mit dem Umgang eines iPad und man sollte den Umgang mit Maus und Tastatur beherrschen.
Die Ergebnisse werden mit durchschnitts Ergebnisse angezeigt in Form eines Linie Diagramm.


\pagebreak

\section{General Conditions and Constraints}
\begin{explanation}
This subject describes the framework conditions to be observed by all stakeholders when the project idea is implemented into concrete measures for realizing the system. Framework conditions, e.g., budget situation, existing know-how, legal provisions, cooperations, commitment to partners and deadlines, may be turned into specifications for project execution.
Technical framework conditions, e.g., development environments and platforms, IT infrastructure, applicable standards and regulations, or specifications of off-the-shelf products, lead to additional (non-functional) requirements for system development.
\end{explanation}

\begin{example}
The proposed system has to the deal with the following constraints:
\begin{itemize}
\item The information about the medical condition of the pupils is strictly confidential.
\item The GUI of the information system must be intuitive.
\item The application must have a small footprint and a local database.
\item A backup concept is mandatory
\item The application is multi-language capable (english and german)
\end{itemize}
\end{example}

\begin{description}
	Alle Daten werden in einer Datenbank gespeichert und ausgewertet.
	Die GUI muss benutzerfreundlich und für Kinder leicht verständlich sein.
	Es soll eine kleine Geschichte geben, damit Kinder keine langeweile bei der Testung haben.
	Da Kinder das Programm verwenden werden, muss das Programm auf deutsch sein. 
	Da wir das Projekt einer 5. Klasse fortsetzen werden und wir erst weitere Aufgaben in ein paar wochen bekommen, müssen wir uns erst um das vorherige Programm Schnapp 4 informieren.

\end{description}

\pagebreak

\section{Project Objectives and System Concepts}
\begin{explanation}
In the Subject Project Objectives and System Concepts, the acquirer describes his vision of a new project or system on a high abstraction level. Project objectives and system concepts may concern several aspects, e.g., the introduction of innovations, the definition of objectives (quality, deadline and cost objectives), the operation of the system in its operating environment and the use of new, improved functionalities.
\end{explanation}

\begin{example}
The project objectives can be summarized as follows:
\begin{itemize}
\item The doctor is documenting the examination results while examining the students
\item Input form assists her/him to input information in a structured and easy way
\item Common situations (need for vaccinations, check for need of dental brace, etc.) are a one-click-job for the doctor
\item Info sheet for parents can be printed right after examination
\item Report is a one-click-job at the end of the day
\end{itemize}
\end{example}



Die Anforderungen an einer aussagekräftigen und gleichzeitig für die schulische Praxis geeignete Lernverlaufsdiagnostik sind hoch. Die Tests sind kurz und geben einen Einblick in den Fortschritts des jeweiligen Schülers.
Eines der größten Hindernisse wird die konstruktion von Paralleltest die die vergangene Kompetenz nochmals abfrägt. Desweiteren wäre eine internetbasierte Realisierung von Vorteil.
Anforderungen:
Jeder Nutzer (Kind) soll einen eigenen Zugang bekommen.
Jeder Lehrer soll einen eigenen Zugang erhalten und so die Kinder kontrollieren können.
Weicht ein Schüler vom Mittelwert ab wird dieser farblich markiert sowohl positiv als auch negativ.
\subsection{Mathe erste Klasse}
Kompetenz basale Vorläuferfähigkeiten (20 Aufgaben):
\begin{itemize}
    \item Mengenvergleich, 2. Bilder als Auswahlmöglichkeit wobei jenes Bild ausgewählt werden muss das mehr Elemente hat. (z.B.: Kreise)
    \item Zahlenvergleich, Größere Zahl von Zwei Zahlen (0-100) auswählen.
    \item Gesprochene Zahl erkennen (0-100) von fünf dargestellten Zahlen. Ausgabe über Kopfhörer bzw.Lautsprecher.

\end{itemize}

Kompetenz fortgeschrittene Vorläuferfähigkeit (17 Aufgaben):
\begin{itemize}
    \item Nachfolger 1er, bei dem zwei Einerschritten aufsteigende / absteigende Zahlen bis 20 dargeboten werden und derjenige Distraktor ausgewählt werden soll der die Reihe (um 1 mehr / weniger) ergänzt.


    \item Nachfolger 2er, bei dem in 2er Schritten dargestellte Reihen verwendet werden
    \item Zahlenstrahl, bei dem eine Zahl (1-20) ausgegeben wird (Kopfhörer), dann soll der Zahlenstrahl mit der ausgewählten Zahl gefunden werden.
\end{itemize}
  Curriculare Kompetenzen (15 Aufgaben Zahlenraum bis 20):
\begin{itemize}
    \item Additionsaufgaben
    \item Subtraktion
  \item Gleichungen, bei dem eine aus Zahlen dargebotene Gleichung die Korrekte Entsprechung mit Würfeln ausgewählt werden soll.
\end{itemize}

\subsection{	Mathematik für 2. Klasse}
Kompetenzbereich Vorläuferkompetenzen (24 Aufgaben)
\begin{itemize}
    \item Zahl erkennen (0-1000)
    \item Zahlvergleich zweier Geldbeträge (-100€) <>= 
    \item Zahlenstrahl (-100)
    \item Formen mit richtiger Spiegelachse erkennen.
\end{itemize}

Curriculare Kompetenzen (28 Aufgaben Zahlenraum bis 100)
\begin{itemize}

    \item Addition
    \item Subtraktion
    \item Multiplikation
    \item Verdoppeln
    \item Halbieren (Vorgegebene Zahl korrekte Halbierte/Verdoppelte Zahl auswählen lassen)
    \item Auf 100 ergänzen (Bsp.: 68+x=100 | x= 32)
\end{itemize}

\subsection{Mathematik 3. Klasse}

Kompetenz Rechnen (20 Aufgaben): 

\begin{itemize}
    \item Addition
    \item Subtraktion
    \item Multiplikation
    \item Division
    \item Erweiterte Zahlenstrahlaufgaben (Zuerst Skala erkannt werden soll und dann die Lösung angegeben werden soll)
\end{itemize}
Kompetenz Geometrie (9 Aufgaben):
\begin{itemize}
    \item Formlegeaufgaben (Kleine Teilstücke gegeben, richtige Figur wird ausgewählt die aus den Teilstücken zusammengesetzt werden könnte.)

    \item Symmetrieaufgabe, verschiedene Landesflaggen gegeben. Es sollte die Flagge ausgewählt werden, die eine Spiegelachse enthält. (Keine Achsen werden eingezeichnet)

    \item Rotationsaufgaben, bei denen diejenige Figur ausgewählt werden soll, die im Vergleich zu einer Referenzfigur gespiegelt wurde (anstatt rotiert).

\end{itemize}
Kompetenz Einheiten (8 Aufgaben):

\begin{itemize}
    \item Umrechenaufgaben: g – kg, cm – m.  


    \item Einheitenwahl, z.B.: 70g oder 70kg für einen ausgewachsenen Menschen

\end{itemize}

\pagebreak
\section{Opportunities and Risks}
\begin{explanation}
The Subject Opportunities and Risks comprises data which are normally prepared in industrial business plans. Frequently, an anonymous market with potential acquirers, which could be interested in the new product or system idea, will be analyzed at first. Therefore, the contents of this subject is characterized by a certain uncertainty or fuzziness. The subjects examines the chances of achieving profit on the market with a specific product or system. In addition to the chances, the risks of failing on the market or sustaining losses with a product or system should be analyzed.
\end{explanation}

\begin{example}
The project has the following opportunities:
\begin{itemize}
\item The doctor is able to increase his time with his patients.
\item The time for bureaucratic work declines.
\item The quality will increase
\end{itemize}

The following risk have to be taken into account.
\begin{itemize}
\item Data transfer of studentsÕ master data from legacy systems is problematic.
\item There is no information about the legacy systems and their data structure.
\item Further there is no information, whether the staff is capable and willing to supply the students master data (names, classes, ...).
\end{itemize}
\end{example}
Möglichkeiten:
\begin{itemize}
    \item Wir können der PH helfen ihre Tests zu optimieren und neue Test einzuführen
    \item Die Tests können spannender für die zu prüfenden Kinder gemacht werden
    Risiken:
    \item Die Datenbank könnte Probleme bereiten und die Prüfungen können verloren gehen, was den Prüfern an der PH zusätzliche Arbeit erbringen könnte
    \item Die Tests könnten während der Prüfung Bugs aufweisen und die Prüfung unbrauchbar machen
    \item Wenn die Zugänge nicht richtig verteilt werden könnten Kinder falsche Ergebnisse bekommen und nicht nach ihrem tatsächlichem Fortschritt gefördert werden
\end{itemize}



\pagebreak
\section{Planning}
\begin{explanation}
The planning specifies the organizational and commercial project execution and system development aspects. The project organization, e.g., matrix organization and steering committees, and the responsibilities for the decision-making processes within project will be specified.
The Project Leader will be appointed, his tasks will be defined. Available resources, funds and specialist personnel will be determined. Start and end date for the project will be specified. The planning can be based on the statements developed in the subject Project Objectives and System Concepts, which makes additional statements on feasibility, funding and schedules.

The following parts must be included:
\begin{itemize}
\item List of major project milestones.
\item Assign project lead and other outstanding roles to team members.
\item Give a rough estimate how many resources you need (human resources, licenses, servers, etc.)
\end{itemize}
\end{explanation}
Answer the following questions when preparing this section:
\begin{itemize}
\item Einarbeiten in die Entwicklungsumgebung und in den Deployment-Workflow
\item Review der derzeit in Arbeit befindlichen Designenhancements
\item Umsetzen von Änderungswünsche (in Absprache mit ph und dem "alten“ Schnapp-Team)
\item Eigene Verbesserungen vornehmen
\end{itemize}
Team Rollen:
\begin{itemize}
\item Leader: Patrick Litzlfellner
\end{itemize}
Ressourcen:
\begin{itemize}
\item Personen: 4 	
\end{itemize}


\end{document}  